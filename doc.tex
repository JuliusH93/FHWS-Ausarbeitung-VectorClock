%% bt: Bachelor Thesis
%% mt: Master Thesis, 'Studienschwerpunkt' will be disabled
\documentclass[ngerman,bt]{dbvdoc}

% Alternative Solutions:
% - use utf8 on your home computer
% - use an editor capable of converting character-sets and editing utf8 files
%   on a latin1 system (some versions of vi do)
% - use \"{a}, \"{o}, \"{u}, \ss{} instead of non-ascii characters

\usepackage[utf8]{inputenc}
%%\usepackage[utf8,latin1]{inputenc}  %% Alternative Eingabe
\usepackage[T1]{fontenc}
\usepackage{amsfonts}
\usepackage{amssymb}
\usepackage{amsmath}
\usepackage[ngerman]{babel}
\usepackage[babel,german=quotes]{csquotes}
\usepackage[backend=biber,maxnames=99,language=german,style=alphabetic,sortcites=true]{biblatex}
\usepackage{hyperref}
\usepackage{breakurl}
%\usepackage[hyphenbreaks]{breakurl}  %% Falls in \url auch bei Bindestrich getrennt werden soll. Weitere Optionen: RTFM
\usepackage{graphicx}
\usepackage{epsfig}
\usepackage{subfigure}
\usepackage{psfrag}
\usepackage{color}

% \setcounter{lotdepth}{2}

% ++ es werden keine underfull hboxes als Fehler ausgegeben,
%    da das ja nur heißt, dass die Seite noch nicht ganz voll ist
\hbadness=10000
\clubpenalty = 10000 % schliesst Schusterjungen aus
\widowpenalty = 10000 % schliesst Hurenkinder aus

\pagenumbering{roman}

%\bibliographystyle{galpha2a}
\renewcommand*{\labelalphaothers}{\textsuperscript{}} %% plus-zeichen abschalten
\bibliography{doc}
\addbibresource{doc.bib}
\addbibresource{lit.bib}

%%%%%%%%%%%%%%
%%% MACROS %%%
%%%%%%%%%%%%%%
%% if macros shall be used, put them into a separate file 'macros.tex'
%%\newcommand{\qq}[1]{\glqq #1\grqq{}}

\newcommand{\vcDash}[1]{\ifthenelse{\equal{#1}{-}}{\text{-}}{#1}}
\newcommand{\vc}[3]{$(\vcDash{#1}, \vcDash{#2}, \vcDash{#3})$}


%%%%%%%%%%%%%%%%%%%%
%%% Worttrennung %%%
%%%%%%%%%%%%%%%%%%%%
%% if hyphenation patterns are needed, put them into a separate file 'hyph.tex'
%%\input{hyph}


%%%%%%%%%%%%%%
%\setcounter{tocdepth}{5}
%\setcounter{secnumdepth}{5}

\sloppy		%% avoid writing over linebreak


\begin{document}
\clearpage
  \selectlanguage{\ngerman}
  \begin{deckblatt}
    \Titel{Titel Arbeit}
    \Name{Mustermann}
    \Vorname{Matthias}
    \Wohnort{Musterstadt}
%    \Geburtsort{Musterstadt}
%    \Geburtsdatum{01.01.1900}
    \BetreuerA{Erstpr\"{u}fer: Prof.~Dr.-Ing.~Frank Deinzer}
    \BetreuerB{Zweitpr\"{u}ferin: Prof.~Dr.-Ing.~Martina Meier}
%    \Start{1. XXX 20XX}
    \Ende{1. YYY 20YY}
    \Fach{Informatik}  % for Master: \Fach{Informationssysteme}
    \Schwerpunkt{Medieninformatik}
    \Angefertigt{Angefertigt an der Fakultät für Informatik und Wirtschaftsinformatik der Hochschule für angewandte Wissenschaften
 Würzburg-Schweinfurt/bei der Firma XYZ}
  \end{deckblatt}
\cleardoublepage

Hiermit versichere ich, dass ich die vorgelegte Bachelorarbeit/Masterarbeit selbstständig verfasst und noch nicht
anderweitig zu Prüfungszwecken vorgelegt habe. Alle benutzten Quellen und Hilfsmittel sind
angegeben, wörtliche und sinngemäße Zitate wurden als solche gekennzeichnet.\\[15mm]
Würzburg, den\\[20mm]
(Unterschrift)

\clearpage

\begin{center}
\bf Übersicht
\end{center}
TEXT DEUTSCH


\vfill
\begin{center}
\bf Abstract
\end{center}
TEXT ENGLISCH

\vfill
\cleardoublepage

\tableofcontents

\cleardoublepage \pagenumbering{arabic}

%%%%%%%%%%%%%%%%%%%%%%%
%%% Inlucde chapter %%%
%%%%%%%%%%%%%%%%%%%%%%%
\include{doc01}   
\cleardoublepage
%\include{doc02}   
%\cleardoublepage
%% usw.
\nocite{*}
%%%%%%%%%%%%%%
%%% Anhang %%%
%%%%%%%%%%%%%%
%\begin{appendix}
%\include{doc-a0} 
%\cleardoublepage
%\include{doc-a1} 
%\cleardoublepage
%%usw...
%\end{appendix}

%% Literatur
\phantomsection
\addcontentsline{toc}{chapter}{\bibname}
\printbibliography
\cleardoublepage

%% Bilderverzeichnis
\phantomsection
\addcontentsline{toc}{chapter}{\listfigurename}
\listoffigures\cleardoublepage

%% Tabellenverzeichnis
\phantomsection
\addcontentsline{toc}{chapter}{\listtablename}
\listoftables\cleardoublepage


\end{document}

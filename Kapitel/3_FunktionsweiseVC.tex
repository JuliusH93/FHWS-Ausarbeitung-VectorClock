\section{Funktionsweise von Vektoruhren}
\subsection{Prinzipielle Funktionsweise}

Wie bereits beschrieben wurde, bilden Vektoruhren eine Erweiterung der Lamportuhr. Dabei wird pro Node im System ein Integer als Zähler für den Node in der Vektoruhr gespeichert. Im Allgemeinen werden durch eine Vektoruhr Zeitstempel mit Events im System assoziiert \cite{Baldoni:2002:FDC:1435723.1437765}[S. 3].

Jeder Node im System besitzt seine eigene Vektoruhr in der Form eines Vektors $VC_i[1...n]$, wobei alle Elemente der Uhr mit $0$ initialisiert werden. Die Uhr eines Nodes wird wie folgt verwaltet und aktualisiert:

\begin{itemize}
	\item[R1]Jedes mal, wenn ein Node $P_i$ ein Event auslöst, muss dieser seine Vektoruhr für den Eintrag $VC_i$ um Eins hochzählen, es gilt  $VC_i[i] := VC_i[i] + 1$. Dadurch verdeutlicht der Node, dass er etwas getan hat und signalisiert dies den anderen Nodes durch aktualisieren seiner Vektoruhr 
	\item[R2]Sendet ein Node $P_i$ eine Nachricht $m$, so hängt er seine aktuelle Vektoruhr $VC_i$ an die zu sendende Nachricht an. Auf diese Weise gelangt die Uhr zu dem Empfänger der Nachricht.
	\item[R3]Empfängt ein Node $P_i$ eine Nachricht, so muss er seine Vektoruhr aktualisieren. Dabei geht er wie folgt vor: $VC_i = \max(VC_i, m.VC)$. Dies bedeutet, dass der Node für jedes Element seiner Vektoruhr überprüft ob der Wert in der Uhr der Nachricht größer als der eigene ist. Sollte dies der Fall sein, wird der eigene Wert an dieser Stelle mit dem Wert aus der anderen Uhr überschrieben.
\end{itemize} \cite{Baldoni:2002:FDC:1435723.1437765}[S. 4]

Die Werte innerhalb einer Vektoruhr $VC_i$ haben eine besondere Bedeutung für den Node $P_i$. $VC_i[i]$ gibt die Anzahl an Events an, welche $P_i$ zu dem Zeitpunkt des Betrachtens verarbeitet hat. Die anderen Werte der Uhr ($VC_i[j]$ mit $j \neq i$) zeigen an, dass alle Events die durch den Node $P_j$ verarbeitet wurden, sich kausal betrachtet in der Vergangenheit von $P_i$ befinden.

Verdeutlicht wird die Verwaltung der Vektoruhren innerhalb eines Systems mit drei Nodes an dem recht simplen Beispiel X.Y.

\subsection{Effizienten Speicherung der Vektoruhren}
\subsection{Umsetzung in C\#}


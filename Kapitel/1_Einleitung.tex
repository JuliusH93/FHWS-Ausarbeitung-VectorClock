\section{Einleitung}
\subsection{Motivation}
Das Network Time Protocol (NTP) ist eines der Ältesten noch immer verwendete Netzwerkprotokolle.
NTP wurde 1985 als RFC 958 veröffentlicht und stellt sicher, dass alle Maschinen innerhalb eines Netzwerkes die selbe Zeitkonfiguration besitzen.
Durch die zentrale Konfiguration und Verteilung der Zeit im Netzwerk wird der Administrationsaufwand deutlich reduziert.
Der Hauptverwendungszweck für NTP ist jedoch ein Anderer: Nämlich das ständige synchronisieren der aktuellen Zeit auf den Systemen.
Dies ist notwendig, da jede Uhr einen Fehler aufweist, der dazu führt, dass ihre Zeit abdriftet.
Daher wird ein Zeitserver im Netzwerk eingesetzt der seine lokale Zeit auf die Clients verteilt.
Dieser synchronisiert sich wiederum mit einer Zeitquelle höherer Qualität wie z.B. einer Atomuhr.

Seit der Veröffentlichung von NTP vor knapp 30 Jahren hat sich die Computerlandschaft deutlich verändert.
Die breite Adaptierung des Internets als weltweite Kommunikationsbasis und die Fragmentierung der Anwenderhardware fordern dezentral arbeitende Systeme.
So besitzt ein typischer Anwender heute neben einen klassischen Computer, ein Smartphone und eventuell ein Tablet.
Trotzdem sollen seine Daten auf allen Geräten gleichermaßen zugreifbar sein.
Um diese Anwenderflexibilität zu ermöglichen, werden heute viele Anwendungen als verteilte Systeme entworfen.

Solche verteilte Anwendungen bringen viele neue Herausforderungen mit sich.


\subsection{Zielsetzung}
\subsection{Aufbau}

    
  
